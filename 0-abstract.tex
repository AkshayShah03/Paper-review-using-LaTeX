%Engaging opener/ ankeiler/ puzzle
The Covid-19 pandemic seems to have cause a rise in conspiracy theories that spread falsehoods, such as that vaccines might contain microchips, or that Covid-19 is a made-up disease. 
Such conspiracy theories can harm trust in government institutions. This is problematic since dealing with crises effectively requires trust in the measures put in place by these institutions.
Previous research has shown that conspiracy beliefs are linked to both social factors and psychological factors. Interestingly, few studies combine these approaches. 
%Clearly recognizable research question
This paper seeks to understand how social and emotional factors shape conspiracy beliefs.
%Approach/methods
The paper
\begin{seriate}
    \item empirically tests what the effects of political orientation and conspiracy beliefs are on trust in government institutions, in the context of this crisis, 
    \item explores a mechanism proposed by some scholars, that societal crises create a psychological predisposition for feelings of lack of control, which then lead to conspiracy beliefs, as a form of sense-making mechanism to cope with the crisis, and
    \item explores the relationship that political orientation has with the previous mechanism. Although studies have shown that people with extreme political views tend to believe more in conspiracies, the current literature misses a study looking into the relation between political orientation and perceived lack of control during crises. I propose that societal crises lead to feelings of lack of control particularly more for the right than for the left. These feelings lead to increased belief in conspiracy theories, and thereby lead to lower trust in government.
\end{seriate}
%Main finding(s)/ take-home message